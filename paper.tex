\documentclass[12pt]{article}
\usepackage[utf8]{inputenc}
\usepackage{graphicx}
\usepackage{amsmath}
\usepackage{hyperref}
\usepackage{cite}

\title{Feasibility Study of a Bioengineered Synaptic Relay Organ for Nerve Regeneration and Neuroprosthetic Integration}
\author{Your Name}
\date{\today}

\begin{document}
	
	\maketitle
	
	\begin{abstract}
		This paper explores the conceptual design and feasibility of a bioengineered relay organ that serves as an adaptable interface between regenerating peripheral nerves and artificial devices. The system includes agnostic relay cells capable of synaptic specialization, spatially organized axonal convergence, and modular interfaces to synthetic components. We follow a unified biological path and explore multiple engineering strategies for the synthetic interface, with special focus on optical, electrical, and chemical transduction methods.
	\end{abstract}
	
	\tableofcontents
	
	\section{Introduction}
	\begin{itemize}
		\item Background on peripheral nerve injuries
		\item Limitations of current nerve repair and neuroprosthetic interfaces
		\item Rationale for a function-agnostic, adaptive relay organ
		\item Goals and scope of this feasibility study
	\end{itemize}
	
	\section{Biological Foundations}
	\subsection{Neuron-Synapse Compatibility}
	\subsection{Design of Relay Cells}
	\subsection{Specialization Mechanisms}
	\subsection{Axonal Sorting and Spatial Organization}
	
	\section{Engineering the Relay Organ}
	\subsection{Structural Design and Encapsulation}
	\subsection{Biocompatibility and Long-term Maintenance}
	\subsection{Support Cell Roles and ECM Design}
	
	\section{Interface Options with Synthetic Devices}
	\subsection{Optical Interface}
	\subsection{Electrical Interface}
	\subsection{Chemical/Fluidic Interface}
	\subsection{Comparative Analysis of Interface Methods}
	
	\section{Feasibility Challenges}
	\subsection{Mechanical Stability and Alignment}
	\subsection{Immunological Barriers}
	\subsection{Scalability and Manufacturability}
	
	\section{Future Directions}
	\begin{itemize}
		\item Prototype development pathways
		\item Key experimental milestones
		\item Long-term vision for clinical translation
	\end{itemize}
	
	\section{Conclusion}
	
	\bibliographystyle{plain}
	\bibliography{references}
	
	
	
\end{document}